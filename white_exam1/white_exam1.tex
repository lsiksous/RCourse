% Created 2020-11-02 Mon 20:06
% Intended LaTeX compiler: pdflatex
\documentclass[11pt]{article}
\usepackage[utf8]{inputenc}
\usepackage[T1]{fontenc}
\usepackage{graphicx}
\usepackage{grffile}
\usepackage{longtable}
\usepackage{wrapfig}
\usepackage{rotating}
\usepackage[normalem]{ulem}
\usepackage{amsmath}
\usepackage{textcomp}
\usepackage{amssymb}
\usepackage{capt-of}
\usepackage{hyperref}
\author{Laurent Siksous}
\date{\today}
\title{White Exam R I}
\hypersetup{
 pdfauthor={Laurent Siksous},
 pdftitle={White Exam R I},
 pdfkeywords={},
 pdfsubject={},
 pdfcreator={Emacs 27.1 (Org mode 9.4)}, 
 pdflang={English}}
\begin{document}

\maketitle

\section*{Description}
\label{sec:org87793da}
The exam will take the form of MCQs. There will be 3 types of questions.

\begin{itemize}
\item questions about R and programming (5p)

\item questions about understanding of R scripts (10p)

\item questions asking you to use R in order to make calculations (5p)
\end{itemize}

\section*{Questions about R}
\label{sec:orge5bf8c8}
\subsection*{Question 1}
\label{sec:org2191168}
Which of the following objects are not a data type of R ?

\begin{itemize}
\item A) Vectors and matrices

\item B) Lists and arrays

\item C) Factors

\item D) Functions
\end{itemize}

\subsection*{Question 2}
\label{sec:org1b74db9}
R functionality is divided into a number of what ?

\begin{itemize}
\item A) Packages

\item B) Domains

\item C) Classes

\item D) All of the above
\end{itemize}

\subsection*{Question 3}
\label{sec:org50584af}
The \uline{\uline{\uline{\uline{\uline{\_\_}}}}} in R is a vector.

\begin{itemize}
\item A) Basic data structure

\item B) Advance data structure

\item C) Basic data types

\item D) None of the Above
\end{itemize}

\subsection*{Question 4}
\label{sec:org9c9848f}
To view a list of preloaded datasets in R, which command do you type into the console ?

\begin{itemize}
\item A) trace()

\item B) data()

\item C) library()

\item B) summary()
\end{itemize}

\subsection*{Question 5}
\label{sec:org17321d5}
“dplyr” is one of the most popular package used in R for manipulating data and
it contains 5 core functions to handle data. Which of the following is not one
of the core functions of dplyr package?

\begin{itemize}
\item A) select()

\item B) filter()

\item C) arrange()

\item D) summary()
\end{itemize}

\section*{Questions about interpretations of R code}
\label{sec:orgd496495}
\subsection*{Question 6}
\label{sec:orgcf25e65}
Which of the following command will remove an R object / variable named “santa” from the workspace?

\begin{itemize}
\item A) remove(santa)

\item B) rm(santa)

\item C) Both

\item D) None
\end{itemize}

\subsection*{Question 7}
\label{sec:orgecd2662}

\begin{center}
\begin{tabular}{lrl}
A & 10 & Sam\\
B & 20 & Peter\\
C & 30 & Harry\\
D & ! & ?\\
E & 50 & Mark\\
\end{tabular}
\end{center}

Missing values in this csv file has been represented by an exclamation mark
(“!”) and a question mark (“?”). Which of the codes below will read the above
csv file correctly into R ?

\begin{itemize}
\item A) csv(‘Dataframe.csv’)

\item B) csv(‘Dataframe.csv’,header=FALSE, sep=’,’,na.strings=c(‘?’))

\item C) csv2(‘Dataframe.csv’,header=FALSE,sep=’,’,na.strings=c(‘?’,’!’))

\item D) dataframe(‘Dataframe.csv’)
\end{itemize}

\subsection*{Question 8}
\label{sec:orgc3d6875}

\begin{center}
\begin{tabular}{rl}
A & B\\
\hline
1 & Right\\
2 & Wrong\\
3 & Wrong\\
4 & Right\\
5 & Right\\
6 & Wrong\\
7 & Wrong\\
8 & Right\\
\end{tabular}
\end{center}

Suppose B is a categorical variable and we wish to draw a boxplot for every
level of the categorical level. Which of the below commands will help us achieve
that?

\begin{itemize}
\item A) boxplot(A,B,data=data)

\item B) boxplot(A\textasciitilde{}B,data=data)

\item C) boxplot(A|B,data=data)

\item D) None of the above
\end{itemize}

\subsection*{Question 9}
\label{sec:org2da4805}
Consider the following function:
\begin{verbatim}
f <- function(x) {
      g <- function(y) {
             y + z
      }

      z <- 4
      x + g(x)
}
\end{verbatim}

If we execute following commands (written below), what would be the output ?

\begin{verbatim}
z <- 10
f(4)
\end{verbatim}

\begin{itemize}
\item A) 12

\item B) 7

\item C) 4

\item D) 16
\end{itemize}

\subsection*{Question 10}
\label{sec:org6dd2521}
What will be the output of following commands?

\begin{verbatim}
A <- paste(“alpha”,”beta”,”gamma”,sep=” ”)
B <- paste(“phi”,”theta”,”zeta”,sep=””)
parts <- strsplit(c(A,B),split=” ”)
parts[[1]][2]
\end{verbatim}

A) alpha

B) beta

C) gamma

D) phi

\subsection*{Question 11}
\label{sec:org3529163}
One of the important phase in a Data Analytics pipeline is univariate analysis
of the features which includes checking for the missing values and the
distribution, etc. we wish to plot histogram for “age”
variable. Which of the following commands will help us perform that task ? 

\begin{itemize}
\item A) hist(data\$age)

\item B) ggplot2::qplot(data\$age,geom=”Histogram”)

\item C) ggplot2::ggplot(data=data,aes(data\$age))+geom\textsubscript{histogram}()

\item D) All of the above
\end{itemize}

\subsection*{Question 12}
\label{sec:orgf2a9889}
Which of the following command will help us to rename the second column in a dataframe named “table” from alpha to beta?

\begin{itemize}
\item A) colnames(table)[2]=’beta’

\item B) colnames(table)[which(colnames==’alpha’)]=’beta’

\item C) setnames(table,’alpha’,’beta’)

\item D) All of the above
\end{itemize}

\subsection*{Question 13}
\label{sec:org93f8996}
We wish to calculate the correlation between “Column2” and “Column3” of a
“dataframe”. Which of the below codes will achieve the purpose ?

\begin{itemize}
\item A) corr(dataframe\$column2,dataframe\$column3)

\item B)
\end{itemize}

(cov(dataframe\$column2,dataframe\$column3))/

(var(dataframe\$column2)*sd(dataframe\$column3))

\begin{itemize}
\item C)
\end{itemize}

(sum(dataframe\$Column2*dataframe\$Column3)-

(sum(dataframe\$Column2)*sum(dataframe\$Column3)/nrow(dataframe)))/

(sqrt((sum(dataframe\$Column2*dataframe\$Column2)-

(sum(dataframe\$Column2)\textsuperscript{3})/nrow(dataframe))*

(sum(dataframe\$Column3*dataframe\$Column3)-

(sum(dataframe\$Column3)\textsuperscript{2})/nrow(dataframe))))

\begin{itemize}
\item D) None of the Above
\end{itemize}

\subsection*{Question 14}
\label{sec:org22dcc57}
Which of the following commands will split the plotting window into 4 X 3
windows and where the plots enter the window column wise ?

\begin{itemize}
\item A) par(split=c(4,3))

\item B) par(mfcol=c(4,3))

\item C) par(mfrow=c(4,3))

\item D) par(col=c(4,3))
\end{itemize}

\subsection*{Question 15}
\label{sec:org0ae011d}
A Dataframe “df” has the following data:

\begin{center}
\begin{tabular}{r}
2017-02-28\\
2017-02-27\\
\ldots{}\\
\end{tabular}
\end{center}

After reading above data, we want the following output:

\begin{center}
\begin{tabular}{l}
28 Tuesday Feb 17\\
27 Monday Feb 17\\
\end{tabular}
\end{center}

Which of the following commands will produce the desired output?

\begin{itemize}
\item A) format(df,”\%d \%A \%b \%y”)

\item B) format(df,”\%D \%A \%b \%y”)

\item C) format(df,”\%D \%a \%B \%Y”)

\item D) None of above
\end{itemize}

\section*{Questions about production of R code}
\label{sec:org4209553}
We use a data set containing data regarding mental health in prison. Every
observation corresponds to an interview conducted with an inmate. The data set
is available in the file : "smp2.csv".

\subsection*{Explications des features de ce dataset}
\label{sec:org7adcca7}

\begin{itemize}
\item l‘âge, age

\item la profession, prof

\item la durée de la peine, quand elle a été prononcée, duree

\item est-ce que le détenu est sous mesure disciplinaire, discip

\item le nombre d'enfants, n.enfant

\item la taille de la fratrie, n.fratrie

\item la variable relative à la scolarisation du détenu qui va de 1 à 5, ecole

\item Est-ce que le détenu a été séparé de sa famille quand il était enfant oui/non,
separation

\item Est-ce qu'il a bénéficié de l'aide d’un juge pour enfants quand il était
enfant, juge.enfant

\item Est-ce qu'il a été placé, place

\item Est-ce qu'il a été victime d'abus, abus

\item la gravité consensuelle, grav.cons
\end{itemize}

Nous retrouvons ensuite les variables diagnostiques,
\begin{itemize}
\item l'existence d'une dépression par le consensus du clinicien, dep.cons

\item un trouble agoraphobique, ago.cons

\item le syndrôme de stress post-traumatique, ptsd.cons
\end{itemize}

L'existence
\begin{itemize}
\item d'un abus d'alcool, alc.cons

\item d'un abus de substances, subst.cons

\item d'une schizophrénie, scs.cons
\end{itemize}

Ensuite nous avons,
\begin{itemize}
\item la variable char qui correspond à un score semi-quantitatif qui évalue
l'importance, l'intensité d'un trouble de la personnalité sous-jacent, char
\end{itemize}

Nous retrouvons les trois dimensions de personnalité
\begin{itemize}
\item recherche de sensation, rs

\item évitement du danger, ed

\item dépendance à la récompense, dr
\end{itemize}

Puis trois variables relatives au risque suicidaire,
\begin{itemize}
\item d'abord un score de risque suicidaire, suicide.s

\item ensuite l'existence d'un haut risque suicidaire, c'est une binarisation de la variable score suicidaire, suicide.hr

\item et enfin l'existence d'antécédents de tentative de suicide, suicide.past
\end{itemize}

Et puis, comme dernière variable nous avons
\begin{itemize}
\item la durée de l'entretien que les enquêteurs ont passée avec le détenu, dur.interv
\end{itemize}

\subsection*{Question 16}
\label{sec:orgdf0ff8f}
How many inmates have more than 3 children ?
\begin{verbatim}
plot(smp.c$age,smp.c$n.enfant)
\end{verbatim}

\begin{center}
\includegraphics[width=.9\linewidth]{/var/folders/lr/51tf4dc1371fb0bcvf3f1gcc0000gp/T/babel-M4doX5/.figureI9Lc1t.png}
\end{center}

\begin{itemize}
\item A) 222

\item B) 55

\item C) 113

\item D) 660
\end{itemize}

\subsection*{Question 17}
\label{sec:org01bec1f}
Estimate the correlation coefficient between the age and the number of children
of detainees ?
\begin{itemize}
\item A) 0.87

\item B) 0.43

\item C) -0.65

\item D) 0.14
\end{itemize}

\subsection*{Question 18}
\label{sec:org525b15a}
Is there a significant correlation between age and sensation research ? Use a
test to show if an older detainee has a lower score of sensation research. What
is the lower end of confidence interval for this test ?
\begin{itemize}
\item A) -0.15

\item B) 0.22

\item C) -0.29

\item D) No significant correlation
\end{itemize}

\subsection*{Question 19}
\label{sec:org7ea0168}
We wish to verify if the interview duration varies whether the detainees have
already tried to kill themselves or not, with the help of a Wilcoxon test. What
is the degree of significance ?

\begin{itemize}
\item A) p ‹ 0.10

\item B) p ‹ 0.05

\item C) p ‹ 0.01

\item D) p ‹ 0.001
\end{itemize}

\subsection*{Question 20}
\label{sec:orgdf6ddc9}
We'd like to predict how the interview duration varies considering several
factors maybe interdependant. To do that, we build a multiple linar regression
model on 4 traits : age, depression, drugs addiction and
schizophrenia. How many minutes should we add or substract to the duration
interview for a detainee with schizophrenia compared to an other ? (regardless of
her age, state of depression or drug consumption)

\begin{itemize}
\item A) 2
\item B) 7
\item C) 5
\item D) I don't know
\end{itemize}
\end{document}