% Created 2020-10-30 Fri 12:21
% Intended LaTeX compiler: pdflatex
\documentclass[11pt]{article}
\usepackage[utf8]{inputenc}
\usepackage[T1]{fontenc}
\usepackage{graphicx}
\usepackage{grffile}
\usepackage{longtable}
\usepackage{wrapfig}
\usepackage{rotating}
\usepackage[normalem]{ulem}
\usepackage{amsmath}
\usepackage{textcomp}
\usepackage{amssymb}
\usepackage{capt-of}
\usepackage{hyperref}
\author{Laurent Siksous}
\date{\today}
\title{Exam R I}
\hypersetup{
 pdfauthor={Laurent Siksous},
 pdftitle={Exam R I},
 pdfkeywords={},
 pdfsubject={},
 pdfcreator={Emacs 27.1.50 (Org mode 9.4)}, 
 pdflang={English}}
\begin{document}

\maketitle
\section*{Description}
\label{sec:org27c1d9f}
The exam will take the form of MCQs. There will be 3 types of questions.
•questions about R and programming (5p)
•questions about understanding of R scripts (10p)
•questions asking you to use R in order to make calculations (5p)

<TBD>
\ldots{} In addition, there will be a bonus of 1 or 2 points for the prediction.
Solutions with a predictionR2between 0.7 and 0.85 will get one additional
point. Solutions with a prediction R2 larger than 0.85 will get two additional
points.
<TBD>

\section*{Questions about R}
\label{sec:orgd29dbd5}
\subsection*{Question 1}
\label{sec:org47837d8}
The function factor() takes as an argument:
\begin{itemize}
\item a sortable vector of character type
\item a boolean
\item a matrix
\item any vector
\end{itemize}
\subsection*{Question 2}
\label{sec:org2881b24}
The function summary()
\begin{itemize}
\item list the environement object and values
\item takes no argument
\item prints statistics of a data frame
\item invokes methods of the class of its first argument
\end{itemize}
\section*{Questions about interpretations of R code}
\label{sec:org5c9b29e}
\subsection*{Question 3}
\label{sec:orgd6da281}
What does this code do ?
\begin{verbatim}
plot(cars$model,cars$price)
\end{verbatim}
\begin{itemize}
\item it creates a graph of two vectors
\item it creates and display a graph of two variables
\item it print two columns of a data set
\item it creates a frequency table of variables
\end{itemize}
\subsection*{Question 4}
\label{sec:org6dd3365}
Which command would return the same value as the following chunk
\begin{verbatim}
range(cars$price)
\end{verbatim}
\begin{itemize}
\item c(min(cars\$price),max(cars\$price))
\item cars[pmin(\$price)][pmax(cars\$price)]
\item cars[which.min(cars\$price),"price"] + cars[which.max(cars\$price),"price"]
\item seq(min(cars\$price),max(cars\$price))
\end{itemize}
\section*{Questions about production of R code}
\label{sec:org2761180}
We use a data set containing mental health in jail data. every observation
corresponds to an interview conducted with an inmate. The data set is available
in the file : "smp2.csv"
\subsection*{Question 5}
\label{sec:orgb334e8c}
How many inmates are employees ?
\begin{itemize}
\item 31
\item 58
\item 90
\item 222
\end{itemize}
\subsection*{Question 6}
\label{sec:org06983ae}
Run an OLS regressions to estimate the model log(dur.interv)= a + b*log(age) + bruit. What is the estimated
value of the coefficient b ? (dur.interv represents the interview duration)
\begin{itemize}
\item -1.712
\item 0.02
\item 0.13
\item 2.35
\end{itemize}
\end{document}